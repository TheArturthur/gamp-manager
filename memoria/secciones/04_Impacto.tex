\chapter{Análisis de impacto}
\label{ch:impacto}

Con respecto a la Agenda 2030 de los Objetivos de Desarrollo Sostenible\cite{desarrollo_sostenible} (ODS) de la Organización de las Naciones Unidas (ONU), vemos que cumplimos las siguientes metas, de las secciones \enquote*{8 - Crecimiento económico}\cite{crecimiento_economico} e \enquote*{9 - Infraestructura}\cite{infraestructura}, respectivamente:
\begin{itemize}
    \item[\textbf{8.2}] 
        \begin{quote}
            Lograr niveles más elevados de productividad económica mediante la diversificación, la modernización tecnológica y la innovación, entre\\ otras cosas centrándose en los sectores con gran valor añadido y un uso intensivo de la mano de obra.
        \end{quote}
    \item[\textbf{9.4}]
        \begin{quote}
            De aquí a 2030, modernizar la infraestructura y reconvertir las industrias para que sean sostenibles, utilizando los recursos con mayor eficacia y promoviendo la adopción de tecnologías y procesos industriales limpios y ambientalmente racionales, y logrando que todos los países tomen medidas de acuerdo con sus capacidades respectivas.
        \end{quote}
\end{itemize}

Esto es, porque consideramos que, a nivel empresarial, el uso de esta aplicación permite aprovechar mejor los recursos dedicados a la monitorización, puesto que requiere de un menor tiempo y dedicación por parte de los trabajadores, resultando en un aumento de la productividad por parte de los mismos. Además, consideramos que la aplicación permite a los usuarios tener una mayor facilidad de uso de los recursos, ya que permite que los mismos sean utilizados de forma más eficiente y eficaz.

Un posible efecto adverso podría ocurri a nivel de recursos pasivos utilizados para la aplicación.

Considerando, por ejemplo, que se quiera tener una Alta Disponibilidad (\textit{High Availability} en inglés), sería necesario alojar la aplicación en múltiples servidores, para poder ser accesible de manera ininterrumpida en caso de fallo de alguno de ellos. También, por ejemplo, sería necesario duplicar todos los recursos utilizados por la aplicación, como serían la base de datos, las interfaces de red, etc. 

Todo esto conllevaría un mayor gasto computacional y energético, lo cual ocasionaría una mayor emisión de gases de efecto invernadero, con el consecuente impacto medioambiental\cite{impacto_medioambiental}.

Sin embargo, también podemos considerar que el impacto medioambiental que pueda surgir de cada uno de los trabajadores realizando la tarea de manera manual, como se ha explicado anteriormente (véase la sección \ref{sec:flujo_manual}), es mayor a este impacto surgido de la aplicación alojada con Alta Disponibilidad.