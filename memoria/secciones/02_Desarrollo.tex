\chapter{Desarrollo}
\label{ch:desarrollo}
En este capítulo vamos a observar los distintos componentes de los que se forma la aplicación. 
\section{Base de datos}
Se ha desarrollado una base de datos en SQLite3\cite{sqlite}, para facilitar la portabilidad de la aplicación entre máquinas durante la fase de desarrollo. Sin embargo, de cara a una implementación productiva, se recomienda portar a una base de datos relacional en SQL.

\subsection*{Diagrama}
El diagrama de la base de datos es el siguiente:
\begin{figure}[h]
    
    \begin{flushleft}
        
        \begin{tikzpicture}[relation/.style={rectangle split, rectangle split parts=#1, rectangle split part align=base, draw, anchor=center, align=center, text height=3mm, text centered}]\hspace*{-0.3cm}
            
            % RELATIONS
            
        \node (projectstitle) {\textbf{Projects}};
        
        \node [relation=3, rectangle split horizontal, rectangle split part fill={lightgray!50}, anchor=north west, below=0.6cm of projectstitle.west, anchor=west] (projects)
        {\underline{IdProject}%
        \nodepart{two}   Name
        \nodepart{three} Datacenter};
        
        \node [below=1.3cm of projects.west, anchor=west] (targets) {\textbf{Targets}};
        
        \node [relation=8, rectangle split horizontal, rectangle split part fill={lightgray!50}, below=0.6cm of targets.west, anchor=west] (target)
        {\underline{IdTarget}%
        \nodepart{two} Name
        \nodepart{three} OS
        \nodepart{four}  Prometheus
        \nodepart{five}  Environment
        \nodepart{six}  Monitoring
        \nodepart{seven}  Alerting
        \nodepart{eight}  Port};

        \node [relation=2, rectangle split horizontal, rectangle split part fill={lightgray!50}, below=0.69cm of target.west, anchor=west] (target_fk)
        {\underline{Exporter\_id}
        \nodepart{two} \underline{Project\_id}};

        \node [below=1.1cm of target_fk.west, anchor=west] (exporters) {\textbf{Exporters}};
        
        \node [relation=4, rectangle split horizontal, rectangle split part fill={lightgray!50}, anchor=north west, below=0.6cm of exporters.west, anchor=west] (exporter)
        {\underline{IdExporter}%
        \nodepart{two}   Name
        \nodepart{three} URL
        \nodepart{four}  Latest\_version};
        
        \node [right=15cm of target.west, anchor=west] (target-project) {1:n};
        \node [right=8.2cm of exporter.west, anchor=west] (target-exporter) {1:n};
        
        % FOREIGN KEYS
        
        \draw[-latex] (target_fk.two south) -- ++(0,-0.2) -| ($(target_fk.two south) + (11.5,0)$) |- ($(projects.one south) + (0.25,-0.50)$) -| ($(projects.one south) + (0.25,0)$);
        
        \draw[-latex] (target_fk.one south) -- ++(0,-0.4) -| ($(target_fk.one south) + (7,-0.4)$) |- ($(exporter.one south) + (0.25,-0.50)$) -| ($(exporter.one south) + (0.25,0)$);
        
    \end{tikzpicture}
\end{flushleft}
\caption{Diagrama relacional de la Base de Datos utilizada}
\label{fig:diagrama_db}
\end{figure}

% \newpage
\section{Frontend}
Para desarrollar la interfaz (GUI) sobre la cual el usuario interactúa con el programa, se utilizó el framework ElectronJS\cite{ElectronJS}. 

\subsection{ElectronJS}
ElectronJS es un \textit{framework} de programación que permite crear aplicaciones de escritorio con tecnología web. Actualmente está siendo usado para aplicaciones como \textit{Visual Studio Code}, \textit{Microsoft Teams} y \textit{Twitch}, entre muchas otras.

Este \textit{framework} se basa en \textit{Chromium} y \textit{Node.js} para construir aplicaciones de escritorio usando \textit{HTML}, \textit{CSS} y \textit{JavaScript}, de manera que su ejecución sea rápida, eficiente y fácilmente configurable.
También permite construir las aplicaciones independientemente de la plataforma y sistema operativo, pudiendo ser ejecutadas en \textit{Windows}, \textit{Linux} y \textit{MacOS}.



% \subsection{Interfaz Gráfica}
% En esta sección observaremos las distintas ventanas que se utilizan en la aplicación y cómmo crear proyectos y añadir	\textit{targets} a los mismos.

\subsection*{Pantalla de inicio}
\label{sec:gui_home}
En esta primera pantalla (véase \hyperref[fig:gui_home]{Figura \ref{fig:gui_home}}), podemos buscar un proyecto en la base de datos. Sólamente si el proyecto fue añadido por la aplicación estará disponible para consulta. En caso de no encontrarlo, mostrará una alerta indicándolo (véase \hyperref[fig:gui_project_not_found]{Figura \ref{fig:gui_project_not_found}}).

\subsection*{Creación de un nuevo proyecto}
\label{sec:gui_create_project}
Para poder crear y añadir un proyecto a la base de datos de la aplicación, se deberán introducir únicamente el nombre del proyecto y, opcionalmente, el \textit{Datacenter} en el que se ubica (véase \hyperref[fig:gui_create_project]{Figura \ref{fig:gui_create_project}}).

\subsection*{Añadir	\textit{targets}}
\label{sec:gui_add_target}
A la hora de añadir nuevos \textit{targets} a la monitorización, elegiremos un proyecto del desplegable, que contiene todos los proyectos presentes en la base de datos (véase \hyperref[fig:gui_add_target_choose_project]{Figura \ref{fig:gui_add_target_choose_project}}). En caso de que el proyecto no esté, nos dará la opción de añadirlo directamente sin tener que cambiar de ventana para ello (véase \hyperref[fig:gui_add_target_project_not_listed]{Figura \ref{fig:gui_add_target_project_not_listed}}).

Una vez seleccionado el proyecto al que pertenece el nuevo \textit{target}, aparecerá una tabla con los campos que se deben rellenar (véase \hyperref[fig:gui_add_target_test]{Figura \ref{fig:gui_add_target_test}}). También aparecerán los \textit{targets} ya presentes en el proyecto, de forma que podamos comprobar fácilmente si un \textit{target} en cuestión ya está introducido y no generar duplicados.

% \newpage
\section{Backend}
Al realizar operaciones en la GUI explicada anteriormente, se realizarán llamadas a un backend en Python, que se encargará de la gestión de la base de datos, la ejecución de los playbooks de Ansible y la configuración de los targets.
\subsection{Docker}
Docker\cite{docker} es una herramienta open source multiplataforma, con la cual se crean contenedores en los que ejecutar y probar la infraestructura, de forma que se pueda ejecutar cualquier pieza de software independientemente al sistema operativo.

\subsubsection{Imágenes}
Las imágenes\cite{docker_image} de Docker son colecciones de instrucciones, directorios y ficheros, a modo de plantilla, sobre la cual se crea un nuevo contenedor. Podrían ser comparables a una imagen de sistema, por ejemplo, de una máquina virtual.

Se pueden crear imágenes nuevas tomando como base otras ya existentes, pudiendo añadir distintas configuraciones que estarán disponibles en los contenedores que se creen a partir de ellas.

\subsubsection{Contenedores}
Los contenedores\cite{docker_container} son instancias ejecutables de una imagen\cite{docker_image} de Docker. Un contenedor consta de:
\begin{itemize}
    \item Una imagen
    \item Un entorno de ejecución
    \item Un conjunto de instrucciones y comandos
\end{itemize}

\subsubsection*{Uso de Docker en este proyecto}

Para este proyecto se han creado cuatro contenedores en Docker:
\begin{itemize}
    \item Uno para funcionar como servidor de Prometheus, que recogerá las métricas de los	\textit{targets} que se usen en la fase de pruebas.
    \item Uno para funcionar como Alertmanager, para configurar avisos y silencios que se requieran y comprobar que las alertas se configuran correctamente mediante la aplicación.
    \item Un tercero, con Grafana, para comprobar los datos recogidos de manera más visual que usando la interfaz propia de Prometheus.
    \item El último, con la aplicación, que la ejecutará y nos permitirá introducir los datos necesarios para cada configuración.
\end{itemize}

Estos contenedores tienen montados los directorios de la aplicación que necesita cada uno para funcionar, de forma que simulan un entorno de ejecución. De esta forma, reducimos las posibilidades de error por encontrarnos en distintas máquinas, y nos abastecemos de una infraestructura básica con la que simular un entorno productivo.

\subsection{User Interface}
% \input{secciones/desarrollo/backend_ui}

\subsection{Ansible}
Ansible\cite{ansible} es una herramienta de automatización, aprovisionamiento y configuración. Utiliza el protocolo SSH (aunque también puede utilizar otros como Kerberos o LDAP si fuera necesario) para realizar operaciones en las máquinas objetivo sin necesidad de tener un agente instalado en ellas.
\subsubsection{Inventario}
Ansible puede puede atacar a distintos nodos o \textit{hosts} al mismo tiempo. Para ello, lo más común es crear un listado de los mismos, agrupándolos según una serie de criterios, como por ejemplo:
\begin{itemize}
    \item Región
    \item Entorno (productivo, test, preproductivo, etc.)
    \item Funcionalidad (servidor web, base de datos, etc.)
\end{itemize}
De esta manera, un mismo \textit{host} puede estar incluido en varios grupos al mismo tiempo.

Además, también pueden especificarse distintas variables, que Ansible usará en la ejecución, tanto para cada \textit{host} como para cada grupo.

\subsubsection{Playbooks}
Los \textit{playbooks} son una forma de automatizar tareas repetitivas en las máquinas. En ellos se especifican una serie de tareas a realizar, y los \textit{hosts} en los que se ejecutará cada una. 

En este proyecto, se ha creado un playbook (ver \hyperref[lst:playbook]{Listing \ref{lst:playbook}}) que se encarga de instalar el agente \textit{node\_exporter} en las máquinas Linux que se especifiquen. Para ello, utiliza un rol de la comunidad de Ansible, que se encarga de realizar la instalación.



\subsection{Exporter}
Una de las funciones de la aplicación es llevar un registro de las versiones de los exporters que hay instaladas en cada máquina, para poder ser actualizados en caso de que hubiera actualizaciones.

Para ello, se ha creado un exporter, siguiendo la guía oficial\cite{exporter_creation}, que añade una métrica a las publicadas por \textit{node\_exporter} llamada \textit{version\_info}, que contiene la versión de cada agente instalado y la versión más reciente disponible.

Esto nos permite enviar alertas cuando un agente queda desactualizado, y tener un registro de las versiones de los agentes instalados en cada máquina.

% \newpage
% \section{Testing}
% \subsection{Virtual Machines}
% % \input{secciones/desarrollo/vm}

% \subsection{Pylint}
% % \input{secciones/desarrollo/pylint}

% \subsection{Pynguin}
% \input{secciones/desarrollo/pynguin}