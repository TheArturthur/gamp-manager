\chapter*{Resumen}
\label{ch:resumen}

Actualmente, a la hora de añadir una máquina objetivo (\textit{target}) a la monitorización \textit{onpremise} con Prometheus, se puede hacer de dos formas distintas:

\begin{itemize}
  \item De manera estática, indicando en un mismo fichero de configuración todos los \textit{targets} de cada proyecto, con las distintas variaciones de etiquetas de cada uno.
  \item De manera dinámica, separando los distintos proyectos en ficheros \url{JSON}, que contienen la lista de \textit{targets} de cada uno.
\end{itemize}

Sin embargo, ambas configuraciones siguen requiriendo una inversión de tiempo bastante grande, ya que se debe añadir manualmente cada uno de los targets a la configuración de Prometheus.

Este proyecto busca solventar esto, usando una aplicación externa, que, según se especifique, genere automáticamente los ficheros JSON necesarios y añada los nuevos targets a la configuración de Prometheus.

\paragraph*{Palabras clave:} prometheus, configuración, automatización

%%%%%%%%%%%%%%%%%%%%%%%%%%%%%%%%%%%%%%%%%%%%%%%%%%%%%%%%%%%
%% Final del resumen. 
%%%%%%%%%%%%%%%%%%%%%%%%%%%%%%%%%%%%%%%%%%%%%%%%%%%%%%%%%%%

%%--------------
\newpage
%%--------------
    
\chapter*{Abstract}
\label{ch:abstract}
In the current time, when adding a host (target) to an on-premise Prometheus monitoring configuration, it can be done in two ways:
\begin{itemize}
  \item Statically, specifying all the targets in a single configuration file, with different variations of tags for each one.
  \item Dynamically, separating the different projects in \url{JSON} files, which contain the list of targets for each one.
\end{itemize}

However, both configurations require a large investment of time, because it is necessary to manually add each one of the targets to the Prometheus configuration.

This projects aims to solve this by using an external application, which, depending on how it is specified, will generate automatically the \url{JSON} files needed and add the new targets to the Prometheus configuration.

\paragraph*{Keywords:} prometheus, configuration, automation