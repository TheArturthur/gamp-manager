\chapter*{Resumen}

A la hora de añadir un \textit{target} a la monitorización \textit{onpremise} con Prometheus, se puede hacer de dos formas distintas:

\begin{itemize}
  \item De manera estática, indicando en un mismo de configuración todos los \textit{targets} de cada proyecto, con las distintas variaciones de etiquetas de cada uno.
  \item De manera dinámica, separando los distintos proyectos en ficheros \url{JSON}, que contienen la lista de \textit{targets} de cada uno.
\end{itemize}

Sin embargo, ambas configuraciones siguen requiriendo una inversión de tiempo bastante grande, ya que se debe añadir manualmente cada uno de los targets a la configuración de Prometheus.

Este proyecto busca solventar esto, usando una aplicación externa, que automáticamente genere los ficheros JSON necesarios y añada los nuevos targets a la configuración de Prometheus.

\paragraph*{Palabras clave:} prometheus, configuración

%%%%%%%%%%%%%%%%%%%%%%%%%%%%%%%%%%%%%%%%%%%%%%%%%%%%%%%%%%%
%% Final del resumen. 
%%%%%%%%%%%%%%%%%%%%%%%%%%%%%%%%%%%%%%%%%%%%%%%%%%%%%%%%%%%

%%--------------
\newpage
%%--------------
    
\chapter*{Abstract}
\todo[inline]{Traducir resumen a inglés}


\paragraph*{Keywords:} prometheus, configuration