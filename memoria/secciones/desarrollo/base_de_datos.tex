Se ha desarrollado una base de datos en SQLite3\cite{sqlite}, para facilitar la portabilidad de la aplicación entre máquinas durante la fase de desarrollo. Sin embargo, de cara a una implementación productiva, se recomienda portar a una base de datos relacional en SQL.
En esta base de datos, alojaremos los nuevos proyectos que se quieran monitorizar, los \textit{targets} asociados al mismo, y los \textit{exporters} utilizados por la aplicación.

\subsection*{Diagrama}
El diagrama de la base de datos es el siguiente:
\begin{figure}[h]
    
    \begin{flushleft}
        
        \begin{tikzpicture}[relation/.style={rectangle split, rectangle split parts=#1, rectangle split part align=base, draw, anchor=center, align=center, text height=3mm, text centered}]\hspace*{-0.3cm}
            
            % RELATIONS
            
        \node (projectstitle) {\textbf{Projects}};
        
        \node [relation=3, rectangle split horizontal, rectangle split part fill={lightgray!50}, anchor=north west, below=0.6cm of projectstitle.west, anchor=west] (projects)
        {\underline{IdProject}%
        \nodepart{two}   Name
        \nodepart{three} Datacenter};
        
        \node [below=1.3cm of projects.west, anchor=west] (targets) {\textbf{Targets}};
        
        \node [relation=8, rectangle split horizontal, rectangle split part fill={lightgray!50}, below=0.6cm of targets.west, anchor=west] (target)
        {\underline{IdTarget}%
        \nodepart{two} Name
        \nodepart{three} OS
        \nodepart{four}  Prometheus
        \nodepart{five}  Environment
        \nodepart{six}  Monitoring
        \nodepart{seven}  Alerting
        \nodepart{eight}  Port};

        \node [relation=2, rectangle split horizontal, rectangle split part fill={lightgray!50}, below=0.69cm of target.west, anchor=west] (target_fk)
        {\underline{Exporter\_id}
        \nodepart{two} \underline{Project\_id}};

        \node [below=1.1cm of target_fk.west, anchor=west] (exporters) {\textbf{Exporters}};
        
        \node [relation=4, rectangle split horizontal, rectangle split part fill={lightgray!50}, anchor=north west, below=0.6cm of exporters.west, anchor=west] (exporter)
        {\underline{IdExporter}%
        \nodepart{two}   Name
        \nodepart{three} URL
        \nodepart{four}  Latest\_version};
        
        \node [right=15cm of target.west, anchor=west] (target-project) {1:n};
        \node [right=8.2cm of exporter.west, anchor=west] (target-exporter) {1:n};
        
        % FOREIGN KEYS
        
        \draw[-latex] (target_fk.two south) -- ++(0,-0.2) -| ($(target_fk.two south) + (11.5,0)$) |- ($(projects.one south) + (0.25,-0.50)$) -| ($(projects.one south) + (0.25,0)$);
        
        \draw[-latex] (target_fk.one south) -- ++(0,-0.4) -| ($(target_fk.one south) + (7,-0.4)$) |- ($(exporter.one south) + (0.25,-0.50)$) -| ($(exporter.one south) + (0.25,0)$);
        
    \end{tikzpicture}
\end{flushleft}
\caption{Diagrama relacional de la Base de Datos utilizada}
\label{fig:diagrama_db}
\end{figure}

\subsection*{Tabla de Proyectos}
En esta tabla, almacenaremos los distintos proyectos que tenemos monitorizados con la aplicación. Cada proyecto tendrá:
\begin{itemize}
    \item \textbf{IdProject}: Un identificador único
    \item \textbf{Name}: El nombre del proyecto
    \item \textbf{Datacenter}: El nombre del datacenter donde se encuentra monitorizado
\end{itemize}

\subsection*{Tabla de Targets}
En esta almacenaremos todos los \textit{targets} que se estén monitorizando. Cada target tendrá:
\begin{itemize}
    \item \textbf{IdTarget}: Un identificador único
    \item \textbf{Name}: El nombre del target
    \item \textbf{OS}: El sistema operativo que utiliza
    \item \textbf{Prometheus}: Si utilizamos el servidor de Prometheus de producción (\url{prod}) o no (\url{nonprod})
    \item \textbf{Environment}: El nombre del entorno al que pertenece
    \item \textbf{Monitoring}: Si se está monitorizando o no
    \item \textbf{Alerting}: Si se está alertando o no
    \item \textbf{Port}: El puerto de la máquina desde el que se recogen las métricas
    \item \textbf{Exporter\_id}: El identificador del exporter que se utiliza para este \textit{target}
    \item \textbf{Project\_id}: El identificador del proyecto al que pertenece
\end{itemize}

\subsection*{Tabla de Exporters}
En esta almacenaremos todos los \textit{exporters} que usemos en la aplicación. Para cada \textit{exporter} tendremos:
\begin{itemize}
    \item \textbf{IdExporter}: Un identificador único
    \item \textbf{Name}: El nombre del exporter
    \item \textbf{URL}: La url del repositorio donde se aloja
    \item \textbf{Latest\_version}: La última versión disponible, para un acceso rápido
\end{itemize}
