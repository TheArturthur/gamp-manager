Docker\cite{docker} es una herramienta open source multiplataforma, con la cual se crean contenedores en los que ejecutar y probar la infraestructura, de forma que se pueda ejecutar cualquier pieza de software independientemente al sistema operativo.

\subsubsection{Imágenes}
Las imágenes\cite{docker_image} de Docker son colecciones de instrucciones, directorios y ficheros, a modo de plantilla, sobre la cual se crea un nuevo contenedor. Podrían ser comparables a una imagen de sistema, por ejemplo, de una máquina virtual.

Se pueden crear imágenes nuevas tomando como base otras ya existentes, pudiendo añadir distintas configuraciones que estarán disponibles en los contenedores que se creen a partir de ellas.

\subsubsection{Contenedores}
Los contenedores\cite{docker_container} son instancias ejecutables de una imagen\cite{docker_image} de Docker. Un contenedor consta de:
\begin{itemize}
    \item Una imagen
    \item Un entorno de ejecución
    \item Un conjunto de instrucciones y comandos
\end{itemize}

\subsubsection*{Uso de Docker en este proyecto}

Para este proyecto se han creado cuatro contenedores en Docker:
\begin{itemize}
    \item Uno para funcionar como servidor de Prometheus, que recogerá las métricas de los	\textit{targets} que se usen en la fase de pruebas.
    \item Uno para funcionar como Alertmanager, para configurar avisos y silencios que se requieran y comprobar que las alertas se configuran correctamente mediante la aplicación.
    \item Un tercero, con Grafana, para comprobar los datos recogidos de manera más visual que usando la interfaz propia de Prometheus.
    \item El último, con la aplicación, que la ejecutará y nos permitirá introducir los datos necesarios para cada configuración.
\end{itemize}

Estos contenedores tienen montados los directorios de la aplicación que necesita cada uno para funcionar, de forma que simulan un entorno de ejecución. De esta forma, reducimos las posibilidades de error por encontrarnos en distintas máquinas, y nos abastecemos de una infraestructura básica con la que simular un entorno productivo.