Ansible\cite{ansible} es una herramienta de automatización, aprovisionamiento y configuración. Utiliza el protocolo SSH (aunque también puede utilizar otros como Kerberos o LDAP si fuera necesario) para realizar operaciones en las máquinas objetivo sin necesidad de tener un agente instalado en ellas.
\subsubsection{Inventario}
Ansible puede puede atacar a distintos nodos o \textit{hosts} al mismo tiempo. Para ello, lo más común es crear un listado de los mismos, agrupándolos según una serie de criterios, como por ejemplo:
\begin{itemize}
    \item Región
    \item Entorno (productivo, test, preproductivo, etc.)
    \item Funcionalidad (servidor web, base de datos, etc.)
\end{itemize}
De esta manera, un mismo \textit{host} puede estar incluido en varios grupos al mismo tiempo.

Además, también pueden especificarse distintas variables, que Ansible usará en la ejecución, tanto para cada \textit{host} como para cada grupo.

\subsubsection{Playbooks}
Los \textit{playbooks} son una forma de automatizar tareas repetitivas en las máquinas. En ellos se especifican una serie de tareas a realizar, y los \textit{hosts} en los que se ejecutará cada una. 

\subsubsection*{Uso de Ansible en este proyecto}
En este proyecto, se ha creado un playbook (ver \hyperref[lst:playbook]{Listing \ref{lst:playbook}}) que se encarga de instalar el agente \textit{node\_exporter} en las máquinas Linux que se especifiquen. Para ello, utiliza un rol de la comunidad de Ansible, que se encarga de realizar la instalación.

